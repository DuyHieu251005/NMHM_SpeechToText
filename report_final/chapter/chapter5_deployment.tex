\chapter{Xây dựng và Triển khai Ứng dụng}

\section{Kiến trúc Hệ thống}

Ứng dụng Vietnamese Speech-to-Text được xây dựng theo kiến trúc client-server với Flask làm backend framework. Hệ thống cho phép người dùng chuyển đổi giọng nói tiếng Việt thành văn bản thông qua 3 mô hình AI khác nhau.

% Chèn sơ đồ kiến trúc hệ thống
% \begin{figure}[H]
%     \centering
%     \includegraphics[width=0.9\textwidth]{img/system_architecture.png}
%     \caption{Kiến trúc tổng thể của hệ thống}
%     \label{fig:system_arch}
% \end{figure}

\subsection{Các thành phần chính}

\begin{enumerate}
    \item \textbf{Frontend (Client):}
    \begin{itemize}
        \item Giao diện web responsive sử dụng HTML5, CSS3, Bootstrap 5
        \item JavaScript xử lý upload file, ghi âm realtime qua Web Audio API
        \item Giao tiếp với backend thông qua REST API (Fetch API)
    \end{itemize}
    
    \item \textbf{Backend (Flask Server):}
    \begin{itemize}
        \item Flask web framework xử lý HTTP requests
        \item API endpoint \texttt{/transcribe} nhận audio và trả về văn bản
        \item Quản lý loading và caching các mô hình AI
    \end{itemize}
    
    \item \textbf{Model Service (AI Models):}
    \begin{itemize}
        \item \textbf{Wav2Vec2}: \texttt{nguyenvulebinh/wav2vec2-base-vietnamese-250h} (checkpoint fine-tuned)
        \item \textbf{PhoWhisper Fine-tuned}: \texttt{vinai/PhoWhisper-small} + LoRA adapter
        \item \textbf{PhoWhisper Base}: \texttt{vinai/PhoWhisper-small} (pre-trained)
        \item \textbf{OpenAI Whisper}: \texttt{openai/whisper-small}
    \end{itemize}
    
    \item \textbf{Audio Processing:}
    \begin{itemize}
        \item FFmpeg (imageio-ffmpeg) để chuyển đổi và resample audio về 16kHz mono
        \item SoundFile để đọc audio đã chuyển đổi
        \item Hỗ trợ định dạng: WAV, MP3, M4A, FLAC, WebM
        \item Kiểm tra silence (RMS energy) trước khi transcribe
    \end{itemize}
\end{enumerate}

\subsection{Cách đóng gói mô hình}

Các mô hình được tải từ Hugging Face Hub và cache locally:
\begin{itemize}
    \item Sử dụng thư viện \texttt{transformers} để load pretrained models
    \item Mô hình được load lazy (chỉ khi cần) để tiết kiệm bộ nhớ
    \item Cache trong RAM để tăng tốc độ inference cho các request tiếp theo
\end{itemize}

\subsection{Luồng xử lý dữ liệu}

\begin{enumerate}
    \item Người dùng upload file audio hoặc ghi âm trực tiếp
    \item Frontend gửi audio data đến endpoint \texttt{/transcribe} qua POST request
    \item Backend nhận file, lưu tạm và xử lý:
    \begin{itemize}
        \item Chuyển đổi audio sang WAV 16kHz mono bằng FFmpeg
        \item Kiểm tra silence (nếu silent thì bỏ qua transcription)
        \item Trích xuất features phù hợp với mô hình được chọn
        \item Chạy inference và decode kết quả
    \end{itemize}
    \item Trả về JSON response chứa văn bản transcription
    \item Frontend hiển thị kết quả cho người dùng
\end{enumerate}

\section{Giao diện và Chức năng}

\subsection{Giao diện Người dùng (UI)}

Giao diện được thiết kế modern, thân thiện với người dùng:

% Chèn hình ảnh giao diện
\begin{figure}[H]
    \centering
    \includegraphics[width=0.9\textwidth]{img/UI.png}
    \caption{Giao diện chính của ứng dụng}
    \label{fig:ui_main}
\end{figure}

\textbf{Các thành phần giao diện:}
\begin{itemize}
    \item \textbf{Header}: Tiêu đề ứng dụng với icon và mô tả ngắn
    \item \textbf{Model Selector}: Dropdown chọn mô hình AI (Wav2Vec2, PhoWhisper Fine-tuned, PhoWhisper Base, Whisper)
    \item \textbf{Tab Navigation}: Chuyển đổi giữa "Tải File" và "Ghi Âm"
    \item \textbf{Upload Zone}: Khu vực drag-and-drop để upload file audio
    \item \textbf{Record Button}: Nút ghi âm với animation và timer
    \item \textbf{Result Box}: Hiển thị kết quả transcription với nút copy
\end{itemize}

\subsection{Các Chức năng Chính}

\begin{enumerate}
    \item \textbf{Chọn Mô hình AI:}
    \begin{itemize}
        \item PhoWhisper Fine-tuned: Mô hình đã fine-tune với LoRA (mặc định)
        \item PhoWhisper Base: Mô hình pre-trained gốc từ VinAI
        \item OpenAI Whisper: Mô hình đa ngôn ngữ mạnh mẽ
        \item Wav2Vec2: Mô hình CTC fine-tuned cho tiếng Việt
    \end{itemize}
    
    \item \textbf{Upload File Audio:}
    \begin{itemize}
        \item Hỗ trợ kéo thả (drag-and-drop)
        \item Hỗ trợ định dạng: WAV, MP3, M4A, FLAC
        \item Giới hạn kích thước: 50MB
        \item Preview audio trước khi chuyển đổi
    \end{itemize}
    
    \item \textbf{Ghi Âm Realtime:}
    \begin{itemize}
        \item Sử dụng Web Audio API và MediaRecorder
        \item Hiển thị timer đếm thời gian ghi
        \item Animation khi đang ghi âm
        \item Nghe lại bản ghi trước khi chuyển đổi
    \end{itemize}
    
    \item \textbf{Chuyển đổi và Hiển thị Kết quả:}
    \begin{itemize}
        \item Hiển thị loading spinner khi đang xử lý
        \item Kết quả hiển thị trong box với badge mô hình
        \item Nút sao chép kết quả vào clipboard
    \end{itemize}
\end{enumerate}

\section{Triển khai}

\subsection{Nền tảng Triển khai}

\textbf{Loại triển khai:} Local (do yêu cầu GPU và dung lượng mô hình lớn)

\subsection{Yêu cầu Hệ thống}

\begin{table}[H]
\centering
\caption{Yêu cầu hệ thống}
\label{tab:system_requirements}
\begin{tabular}{|l|l|}
\hline
\textbf{Thành phần} & \textbf{Yêu cầu} \\
\hline
Python & 3.8 trở lên \\
\hline
RAM & Tối thiểu 8GB (khuyến nghị 16GB) \\
\hline
GPU & NVIDIA GPU với CUDA (khuyến nghị, không bắt buộc) \\
\hline
Dung lượng & ~5GB cho các mô hình \\
\hline
Hệ điều hành & Windows, Linux, macOS \\
\hline
\end{tabular}
\end{table}

\subsection{Cấu trúc Thư mục}

\begin{lstlisting}[style=mystyle]
app/
    app.py              # Flask backend
    requirements.txt    # Dependencies  
    README.md          # Huong dan
    templates/
        index.html     # Giao dien web
\end{lstlisting}

\subsection{Hướng dẫn Cài đặt và Chạy}

\begin{enumerate}
    \item \textbf{Clone repository:}
    \begin{lstlisting}[style=mystyle]
git clone https://github.com/DuyHieu251005/NMHM_SpeechToText.git
cd NMHM_SpeechToText/app
    \end{lstlisting}
    
    \item \textbf{Tạo môi trường ảo:}
    \begin{lstlisting}[style=mystyle]
python -m venv venv

# Windows
venv\Scripts\activate

# Linux/Mac
source venv/bin/activate
    \end{lstlisting}
    
    \item \textbf{Cài đặt dependencies:}
    \begin{lstlisting}[style=mystyle]
pip install -r requirements.txt
    \end{lstlisting}
    
    \item \textbf{Chạy ứng dụng:}
    \begin{lstlisting}[style=mystyle]
python app.py
    \end{lstlisting}
    
    \item \textbf{Truy cập ứng dụng:}
    
    Mở trình duyệt và truy cập: \url{http://localhost:5000}
\end{enumerate}

\subsection{API Endpoints}

\begin{table}[H]
\centering
\caption{Danh sách API Endpoints}
\label{tab:api_endpoints}
\begin{tabular}{|l|l|l|}
\hline
\textbf{Method} & \textbf{Endpoint} & \textbf{Mô tả} \\
\hline
GET & / & Trang chủ (giao diện web) \\
\hline
POST & /transcribe & Chuyển đổi audio thành văn bản \\
\hline
GET & /health & Kiểm tra trạng thái server \\
\hline
\end{tabular}
\end{table}

\subsection{Thư viện và Dependencies}

\begin{table}[H]
\centering
\caption{Các thư viện chính sử dụng}
\label{tab:dependencies}
\begin{tabular}{|l|l|}
\hline
\textbf{Thư viện} & \textbf{Mục đích} \\
\hline
Flask & Web framework \\
\hline
PyTorch & Deep learning framework \\
\hline
Transformers & Load pretrained models từ Hugging Face \\
\hline
imageio-ffmpeg & Chuyển đổi và resample audio \\
\hline
SoundFile & Đọc file audio WAV \\
\hline
PEFT & LoRA adapters cho PhoWhisper \\
\hline
\end{tabular}
\end{table}
