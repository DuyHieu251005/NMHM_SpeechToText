
\chapter{Giới thiệu bài toán}
Trong lĩnh vực Xử lý Ngôn ngữ Tự nhiên (NLP) và Xử lý Tín hiệu, dự án này giải quyết bài toán \textbf{Nhận dạng Tiếng nói Tiếng Việt Tự động (Vietnamese Automatic Speech Recognition - ASR)}. Mục tiêu chính là xây dựng một hệ thống học máy có khả năng chuyển đổi tín hiệu giọng nói của con người thành dạng văn bản tương ứng với độ chính xác cao.

Một cách hình thức, bài toán ASR có thể được định nghĩa là tìm chuỗi từ có khả năng xảy ra cao nhất $\hat{W}$ với đầu vào là chuỗi các đặc trưng âm thanh $X$. Bài toán này được mô hình hóa dưới dạng bài toán phân loại xác suất, trong đó chúng ta tìm cách tối đa hóa xác suất hậu nghiệm:

\begin{equation}
    \hat{W} = \operatorname*{argmax}_{W} P(W|X)
\end{equation}

Trong đó:
\begin{itemize}
    \item $X = \{x_1, x_2, \dots, x_T\}$ đại diện cho chuỗi các vector đặc trưng âm thanh được trích xuất từ đầu vào âm thanh thô (ví dụ: MFCC hoặc Spectrogram).
    \item $W = \{w_1, w_2, \dots, w_N\}$ đại diện cho chuỗi các từ trong câu mục tiêu.
    \item $P(W|X)$ là xác suất của chuỗi từ $W$ khi biết quan sát âm thanh $X$.
\end{itemize}
\vfil